\documentclass[10pt,landscape]{article}
\usepackage{multicol}
\usepackage{calc}
\usepackage{ifthen}
\usepackage[landscape]{geometry}
\usepackage{hyperref}
\setlength{\columnseprule}{0.4pt}

% To make this come out properly in landscape mode, do one of the following
% 1.
%  pdflatex latexsheet.tex
%
% 2.
%  latex latexsheet.tex
%  dvips -P pdf  -t landscape latexsheet.dvi
%  ps2pdf latexsheet.ps


% If you're reading this, be prepared for confusion.  Making this was
% a learning experience for me, and it shows.  Much of the placement
% was hacked in; if you make it better, let me know...


% 2008-04
% Changed page margin code to use the geometry package. Also added code for
% conditional page margins, depending on paper size. Thanks to Uwe Ziegenhagen
% for the suggestions.

% 2006-08
% Made changes based on suggestions from Gene Cooperman. <gene at ccs.neu.edu>


% To Do:
% \listoffigures \listoftables
% \setcounter{secnumdepth}{0}


% This sets page margins to .5 inch if using letter paper, and to 1cm
% if using A4 paper. (This probably isn't strictly necessary.)
% If using another size paper, use default 1cm margins.
\ifthenelse{\lengthtest { \paperwidth = 11in}}
	{ \geometry{top=.5in,left=.5in,right=.5in,bottom=.5in} }
	{\ifthenelse{ \lengthtest{ \paperwidth = 297mm}}
		{\geometry{top=1cm,left=1cm,right=1cm,bottom=1cm} }
		{\geometry{top=1cm,left=1cm,right=1cm,bottom=1cm} }
	}

% Turn off header and footer
\pagestyle{empty}
 

% Redefine section commands to use less space
\makeatletter
\renewcommand{\section}{\@startsection{section}{1}{0mm}%
                                {-1ex plus -.5ex minus -.2ex}%
                                {0.5ex plus .2ex}%x
                                {\normalfont\large\bfseries}}
\renewcommand{\subsection}{\@startsection{subsection}{2}{0mm}%
                                {-1explus -.5ex minus -.2ex}%
                                {0.5ex plus .2ex}%
                                {\normalfont\normalsize\bfseries}}
\renewcommand{\subsubsection}{\@startsection{subsubsection}{3}{0mm}%
                                {-1ex plus -.5ex minus -.2ex}%
                                {1ex plus .2ex}%
                                {\normalfont\small\bfseries}}
\makeatother

% Define BibTeX command
\def\BibTeX{{\rm B\kern-.05em{\sc i\kern-.025em b}\kern-.08em
    T\kern-.1667em\lower.7ex\hbox{E}\kern-.125emX}}

% Don't print section numbers
\setcounter{secnumdepth}{0}


\setlength{\parindent}{0pt}
\setlength{\parskip}{0pt plus 0.5ex}


% -----------------------------------------------------------------------

\begin{document}

\raggedright
\footnotesize
\begin{multicols}{3}


% multicol parameters
% These lengths are set only within the two main columns
%\setlength{\columnseprule}{0.25pt}
\setlength{\premulticols}{1pt}
\setlength{\postmulticols}{1pt}
\setlength{\multicolsep}{1pt}
\setlength{\columnsep}{2pt}

\begin{center}
     \Large{\textbf{CS2040S Cheatsheet}} \\
     \small{Made by g-tejas. Last updated: \today} \\
     \rule{1\linewidth}{0.5pt}
\end{center}

\section{Algorithm Analysis}
Add the recurrence relation table from 1101S here (maybe can find on luminus)
\subsection{Common \texttt{documentclass} options}
\newlength{\MyLen}
\settowidth{\MyLen}{\texttt{letterpaper}/\texttt{a4paper} \ }
\begin{tabular}{@{}p{\the\MyLen}%
                @{}p{\linewidth-\the\MyLen}@{}}
\texttt{10pt}/\texttt{11pt}/\texttt{12pt} & Font size. \\
\texttt{letterpaper}/\texttt{a4paper} & Paper size. \\
\texttt{twocolumn} & Use two columns. \\
\texttt{twoside}   & Set margins for two-sided. \\
\texttt{landscape} & Landscape orientation.  Must use
                     \texttt{dvips -t landscape}. \\
\texttt{draft}     & Double-space lines.
\end{tabular}

Usage:
\verb!\documentclass[!\textit{opt,opt}\verb!]{!\textit{class}\verb!}!.


\subsection{Packages}
\settowidth{\MyLen}{\texttt{multicol} }
\begin{tabular}{@{}p{\the\MyLen}%
                @{}p{\linewidth-\the\MyLen}@{}}
%\begin{tabular}{@{}ll@{}}
\texttt{fullpage}  &  Use 1 inch margins. \\
\texttt{anysize}   &  Set margins: \verb!\marginsize{!\textit{l}%
                        \verb!}{!\textit{r}\verb!}{!\textit{t}%
                        \verb!}{!\textit{b}\verb!}!.            \\
\texttt{multicol}  &  Use $n$ columns: 
                        \verb!\begin{multicols}{!$n$\verb!}!.   \\
\texttt{latexsym}  &  Use \LaTeX\ symbol font. \\
\texttt{graphicx}  &  Show image:
                        \verb!\includegraphics[width=!%
                        \textit{x}\verb!]{!%
                        \textit{file}\verb!}!. \\
\texttt{url}       & Insert URL: \verb!\url{!%
                        \textit{http://\ldots}%
                        \verb!}!.
\end{tabular}

Use before \verb!\begin{document}!. 
Usage: \verb!\usepackage{!\textit{package}\verb!}!


\subsection{Title}
\settowidth{\MyLen}{\texttt{.author.text.} }
\begin{tabular}{@{}p{\the\MyLen}%
                @{}p{\linewidth-\the\MyLen}@{}}
\verb!\author{!\textit{text}\verb!}! & Author of document. \\
\verb!\title{!\textit{text}\verb!}!  & Title of document. \\
\verb!\date{!\textit{text}\verb!}!   & Date. \\
\end{tabular}

These commands go before \verb!\begin{document}!.  The declaration
\verb!\maketitle! goes at the top of the document.

\subsection{Miscellaneous}
\settowidth{\MyLen}{\texttt{.pagestyle.empty.} }
\begin{tabular}{@{}p{\the\MyLen}%
                @{}p{\linewidth-\the\MyLen}@{}}
\verb!\pagestyle{empty}!     &   Empty header, footer
                                 and no page numbers. \\
\verb!\tableofcontents!      &   Add a table of contents here. \\

\end{tabular}



\section{Document structure}
\begin{multicols}{2}
\verb!\part{!\textit{title}\verb!}!  \\
\verb!\chapter{!\textit{title}\verb!}!  \\
\verb!\section{!\textit{title}\verb!}!  \\
\verb!\subsection{!\textit{title}\verb!}!  \\
\verb!\subsubsection{!\textit{title}\verb!}!  \\
\verb!\paragraph{!\textit{title}\verb!}!  \\
\verb!\subparagraph{!\textit{title}\verb!}!
\end{multicols}
{\raggedright
Use \verb!\setcounter{secnumdepth}{!$x$\verb!}! suppresses heading
numbers of depth $>x$, where \verb!chapter! has depth 0.
Use a \texttt{*}, as in \verb!\section*{!\textit{title}\verb!}!,
to not number a particular item---these items will also not appear
in the table of contents.
}

\subsection{Text environments}
\settowidth{\MyLen}{\texttt{.begin.quotation.}}
\begin{tabular}{@{}p{\the\MyLen}%
                @{}p{\linewidth-\the\MyLen}@{}}
\verb!\begin{comment}!    &  Comment (not printed). Requires \texttt{verbatim} package. \\
\verb!\begin{quote}!      &  Indented quotation block. \\
\verb!\begin{quotation}!  &  Like \texttt{quote} with indented paragraphs. \\
\verb!\begin{verse}!      &  Quotation block for verse.
\end{tabular}

\subsection{Lists}
\settowidth{\MyLen}{\texttt{.begin.description.}}
\begin{tabular}{@{}p{\the\MyLen}%
                @{}p{\linewidth-\the\MyLen}@{}}
\verb!\begin{enumerate}!        &  Numbered list. \\
\verb!\begin{itemize}!          &  Bulleted list. \\
\verb!\begin{description}!      &  Description list. \\
\verb!\item! \textit{text}      &  Add an item. \\
\verb!\item[!\textit{x}\verb!]! \textit{text}
                                &  Use \textit{x} instead of normal
                        bullet or number.  Required for descriptions. \\
\end{tabular}




\subsection{References}
\settowidth{\MyLen}{\texttt{.pageref.marker..}}
\begin{tabular}{@{}p{\the\MyLen}%
                @{}p{\linewidth-\the\MyLen}@{}}
\verb!\label{!\textit{marker}\verb!}!   &  Set a marker for cross-reference, 
                          often of the form \verb!\label{sec:item}!. \\
\verb!\ref{!\textit{marker}\verb!}!   &  Give section/body number of marker. \\
\verb!\pageref{!\textit{marker}\verb!}! &  Give page number of marker. \\
\verb!\footnote{!\textit{text}\verb!}!  &  Print footnote at bottom of page. \\
\end{tabular}


\subsection{Floating bodies}
\settowidth{\MyLen}{\texttt{.begin.equation..place.}}
\begin{tabular}{@{}p{\the\MyLen}%
                @{}p{\linewidth-\the\MyLen}@{}}
\verb!\begin{table}[!\textit{place}\verb!]!     &  Add numbered table. \\
\verb!\begin{figure}[!\textit{place}\verb!]!    &  Add numbered figure. \\
\verb!\begin{equation}[!\textit{place}\verb!]!  &  Add numbered equation. \\
\verb!\caption{!\textit{text}\verb!}!           &  Caption for the body. \\
\end{tabular}

The \textit{place} is a list valid placements for the body.  \texttt{t}=top,
\texttt{h}=here, \texttt{b}=bottom, \texttt{p}=separate page, \texttt{!}=place even if ugly.  Captions and label markers should be within the environment.

%---------------------------------------------------------------------------

\section{Text properties}

\subsection{Font face}
\newcommand{\FontCmd}[3]{\PBS\verb!\#1{!\textit{text}\verb!}!  \> %
                         \verb!{\#2 !\textit{text}\verb!}! \> %
                         \#1{#3}}
\begin{tabular}{@{}l@{}l@{}l@{}}
\textit{Command} & \textit{Declaration} & \textit{Effect} \\
\verb!\textrm{!\textit{text}\verb!}!                    & %
        \verb!{\rmfamily !\textit{text}\verb!}!               & %
        \textrm{Roman family} \\
\verb!\textsf{!\textit{text}\verb!}!                    & %
        \verb!{\sffamily !\textit{text}\verb!}!               & %
        \textsf{Sans serif family} \\
\verb!\texttt{!\textit{text}\verb!}!                    & %
        \verb!{\ttfamily !\textit{text}\verb!}!               & %
        \texttt{Typewriter family} \\
\verb!\textmd{!\textit{text}\verb!}!                    & %
        \verb!{\mdseries !\textit{text}\verb!}!               & %
        \textmd{Medium series} \\
\verb!\textbf{!\textit{text}\verb!}!                    & %
        \verb!{\bfseries !\textit{text}\verb!}!               & %
        \textbf{Bold series} \\
\verb!\textup{!\textit{text}\verb!}!                    & %
        \verb!{\upshape !\textit{text}\verb!}!               & %
        \textup{Upright shape} \\
\verb!\textit{!\textit{text}\verb!}!                    & %
        \verb!{\itshape !\textit{text}\verb!}!               & %
        \textit{Italic shape} \\
\verb!\textsl{!\textit{text}\verb!}!                    & %
        \verb!{\slshape !\textit{text}\verb!}!               & %
        \textsl{Slanted shape} \\
\verb!\textsc{!\textit{text}\verb!}!                    & %
        \verb!{\scshape !\textit{text}\verb!}!               & %
        \textsc{Small Caps shape} \\
\verb!\emph{!\textit{text}\verb!}!                      & %
        \verb!{\em !\textit{text}\verb!}!               & %
        \emph{Emphasized} \\
\verb!\textnormal{!\textit{text}\verb!}!                & %
        \verb!{\normalfont !\textit{text}\verb!}!       & %
        \textnormal{Document font} \\
\verb!\underline{!\textit{text}\verb!}!                 & %
                                                        & %
        \underline{Underline}
\end{tabular}

The command (t\textit{tt}t) form handles spacing better than the
declaration (t{\itshape tt}t) form.

\subsection{Font size}

These are declarations and should be used in the form
\verb!{\small! \ldots\verb!}!, or without braces to affect the entire
document.


\subsection{Verbatim text}

\subsection{Justification}

\subsection{Miscellaneous}

\section{Text-mode symbols}

\subsection{Symbols}

\subsection{Accents}

\subsection{Delimiters}

\subsection{Dashes}

\subsection{Line and page breaks}

\subsection{Miscellaneous}

\section{Tabular environments}

\subsection{\texttt{tabbing} environment}

\subsection{\texttt{tabular} environment}

\subsubsection{\texttt{tabular} column specification}

\subsubsection{\texttt{tabular} elements}

\section{Math mode}

\subsection{Math-mode symbols}

\section{Bibliography and citations}
When using \BibTeX, you need to run \texttt{latex}, \texttt{bibtex},
and \texttt{latex} twice more to resolve dependencies.

\subsection{Citation types}
\settowidth{\MyLen}{\texttt{.shortciteN.key..}}
\begin{tabular}{@{}p{\the\MyLen}@{}p{\linewidth-\the\MyLen}@{}}
\verb!\cite{!\textit{key}\verb!}!       &
        Full author list and year. (Watson and Crick 1953) \\
\verb!\citeA{!\textit{key}\verb!}!      &
        Full author list. (Watson and Crick) \\
\verb!\citeN{!\textit{key}\verb!}!      &
        Full author list and year. Watson and Crick (1953) \\
\verb!\shortcite{!\textit{key}\verb!}!  &
        Abbreviated author list and year. ? \\
\verb!\shortciteA{!\textit{key}\verb!}! &
        Abbreviated author list. ? \\
\verb!\shortciteN{!\textit{key}\verb!}! &
        Abbreviated author list and year. ? \\
\verb!\citeyear{!\textit{key}\verb!}!   &
        Cite year only. (1953) \\
\end{tabular}

All the above have an \texttt{NP} variant without parentheses;
Ex. \verb!\citeNP!.


\subsection{\BibTeX\ entry types}

\subsection{\BibTeX\ fields}

\subsection{Common \BibTeX\ style files}

\subsection{\BibTeX\ example}

\section{Sample \LaTeX\ document}

\rule{0.3\linewidth}{0.25pt}
\scriptsize

No Copyright \copyright\ 2023 Tejas Garrepally

\href{http://wch.github.io/latexsheet/}{http://wch.github.io/latexsheet/}


\end{multicols}
\end{document}
